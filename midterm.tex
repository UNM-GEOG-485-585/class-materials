\documentclass[]{article}
\usepackage[T1]{fontenc}
\usepackage{lmodern}
\usepackage{amssymb,amsmath}
\usepackage{ifxetex,ifluatex}
\usepackage{fixltx2e} % provides \textsubscript
\usepackage{booktabs}

% Added by KB to handle generation of tables with tiny font
\usepackage{floatrow}
\DeclareFloatFont{tiny}{\tiny}% "scriptsize" is defined by floatrow, "tiny" not
\floatsetup[table]{font=tiny}


\providecommand{\tightlist}{%
  \setlength{\itemsep}{0pt}\setlength{\parskip}{0pt}}

% use upquote if available, for straight quotes in verbatim environments
\IfFileExists{upquote.sty}{\usepackage{upquote}}{}
\ifnum 0\ifxetex 1\fi\ifluatex 1\fi=0 % if pdftex
  \usepackage[utf8]{inputenc}
\else % if luatex or xelatex
  \ifxetex
    \usepackage{mathspec}
    \usepackage{xltxtra,xunicode}
  \else
    \usepackage{fontspec}
  \fi
  \defaultfontfeatures{Mapping=tex-text,Scale=MatchLowercase}
  \newcommand{\euro}{€}
\fi
% use microtype if available
\IfFileExists{microtype.sty}{\usepackage{microtype}}{}
\ifxetex
  \usepackage[setpagesize=false, % page size defined by xetex
              unicode=false, % unicode breaks when used with xetex
              xetex]{hyperref}
\else
  \usepackage[unicode=true]{hyperref}
\fi
\hypersetup{breaklinks=true,
            bookmarks=true,
            pdfauthor={},
            pdftitle={},
            colorlinks=true,
            citecolor=blue,
            urlcolor=blue,
            linkcolor=magenta,
            pdfborder={0 0 0}}
\urlstyle{same}  % don't use monospace font for urls
\setlength{\parindent}{0pt}
\setlength{\parskip}{6pt plus 2pt minus 1pt}
\setlength{\emergencystretch}{3em}  % prevent overfull lines

\oddsidemargin = 0pt
\evensidemargin = 0pt
\topmargin = -36pt
\textwidth = 468pt
\textheight = 648pt

\setcounter{secnumdepth}{0}

\author{}
\date{}

\begin{document}

\section{GEOG 485L/585L Midterm Exam}\label{geog-485l585l-midterm-exam}

\subsection{Due Friday, March 11, 2016 before 5:00
pm}\label{due-friday-march-11-2016-before-500-pm}

\emph{Late Exams (determined by when the exam page is posted to your
GitHub repo) will be penalized 10 points (10\%) for each 24-hours or
fraction thereof}

Just as you have done for your milestones and deep-dives, create a web
page with your answers to the exam questions and link to the page from
your homepage (index.html) in GitHub.

Make sure to \emph{clearly format} your writeup so that your answer's
are understandable.

100 pts

\begin{enumerate}
\item
  Add a Google Map to your writeup that has the following
  characteristics (34 pts total):

  \begin{itemize}
  \tightlist
  \item
    Based on the Satellite base map (10 pts)
  \item
    Centered at 36.060574,-107.961692 (5 pts)
  \item
    400 px wide by 300 px high (5 pts)
  \item
    Zoom level 18 (5 pts)
  \item
    With a single marker positioned at the center point (5 pts)
  \item
    With an InfoWindow (with content of your choice) that is displayed
    when the user clicks on it (4 pts)
  \end{itemize}
\item
  What combination of OGC Service and Request
  (e.g.~SERVICE=WMS\&REQUEST=GetFeatureInfo) would you use to perform
  the following? (3 pts each)

  \begin{enumerate}
  \def\labelenumii{\alph{enumii}.}
  \tightlist
  \item
    Determine the spatial extent of an available layer from a Web Map
    Service.
  \item
    Obtain a list of coverages from a Web Coverage service.
  \item
    Determine what file formats are supported by a Web Feature Service
    for the delivery of available data types (i.e.~layers)
  \item
    Retrieve a map image from a Web Map Service
  \item
    Retrieve data from an available coverage from a Web Coverage
    Service

  \end{enumerate}
\item
  Perform the following WMS GetCapabilities request
  (\href{http://neowms.sci.gsfc.nasa.gov/wms/wms?version=1.1.1\&service=WMS\&request=GetCapabilities}{Link\footnote{\texttt{http://neowms.sci.gsfc.nasa.gov/wms/wms?version=1.1.1\&service=WMS\&request=GetCapabilities}}})
  and answer the following questions (3 pts each)

  \begin{enumerate}
  \def\labelenumii{\alph{enumii}.}
  \tightlist
  \item
    What is the name of the service?
  \item
    What file formats are supported by the GetMap request?
  \item
    What are the \emph{names} of three of the layers included in the
    service?
  \end{enumerate}
\item
  Compose a GetMap request for the WMS referenced in the previous
  question that includes the following characteristics. Include in your
  answer both the complete WMS GetMap request and the resulting map
  image that is returned. (15 pts)

  \begin{itemize}
  \tightlist
  \item
    JPEG image format
  \item
    1200 pixels wide (you will need to calculate the height based upon
    the aspect ratio of the bounding box)
  \item
    Bounding Box (EPSG:4326):
    \texttt{Min\ X\ =\ -128\ East\ Longitude,\ Min\ Y\ =\ 21.5\ North\ Latitude,\ Max\ X\ =\ -62\ East\ Longitude,\ Max\ Y\ =\ 54.5\ North\ Latitude}
  \item
    Layer to be mapped = ``MOD\_LSTD\_CLIM\_M''
  \end{itemize}
\item
  From the XML GetCapabilities returned by the following WFS request
  (\href{http://services.nationalmap.gov/arcgis/services/WFS/transportation/MapServer/WFSServer?request=GetCapabilities\&service=WFS}{Link\footnote{\texttt{http://services.nationalmap.gov/arcgis/services/WFS/transportation/MapServer/WFSServer?\ request=GetCapabilities\&service=WFS}}})
  answer the following questions (3 pts each)


  \begin{enumerate}
  \def\labelenumii{\alph{enumii}.}
  \tightlist
  \item
    What is the title for this service?
  \item
    What file format(s) are supported by this service's GetFeature
    request?
  \item
    What is the DefaultSRS or the FeatureType named
    ``WFS\_transportation:Interstate''?
  \item
    What is the WGS84BoundingBox of the FeatureType named
    ``WFS\_transportation:Interstate''?
  \end{enumerate}
\item
  From the XML GetCapabilities returned by the following WCS request
  (\href{http://sdf.ndbc.noaa.gov/thredds/wcs/hfradar_uswc_500m?request=GetCapabilities\&version=1.0.0\&service=WCS}{Link\footnote{\texttt{http://sdf.ndbc.noaa.gov/thredds/wcs/hfradar\_uswc\_500m?request=GetCapabilities\&version=1.0.0\&service=WCS}}})
  answer the following questions (3 pts each)

  \begin{enumerate}
  \def\labelenumii{\alph{enumii}.}
  \tightlist
  \item
    What is the description of the \texttt{v} coverage?
  \item
    How many coverages are available from this service?
  \end{enumerate}
\item
  Formulate a complete DescribeCoverage request for the \texttt{v}
  coverage for the WCS service referenced in the previous question and
  provide a link in your writeup for that request (i.e.~I should be able
  to click the link and get the XML returned in response to your well
  formed request). (3 pts)
\item
  From the returned XML document from the DescribeCoverage request in
  the previous question answer the following questions. (3 pts each)

  \begin{enumerate}
  \def\labelenumii{\alph{enumii}.}
  \tightlist
  \item
    What is the spatial domain for the `v` coverage?
  \item
    What file formats are available for `v` data delivered by this
    service?
  \item
    What SRS(s) are supported by this service for requested data
    delivery?
  \end{enumerate}
\end{enumerate}

\end{document}
